En este trabajo práctico abordamos el desarrollo de herramientas de diagnóstico de red, con el objetivo principal de analizar el protocolo ARP (\emph{Address Resolution Protocol}). Este provee un método para asociar los identificadores de capa enlace y de capa de red de los dispositivos conectados a una LAN (\emph{Local Area Network}); por ejemplo, vincular direcciones MAC (capa enlace) con direcciones IPv4 (capa de red).

Las herramientas desarrolladas\footnote{\emph{arp\_listener\_tool.py}, \emph{arp\_analysis\_tool.py}} permiten capturar pasivamente los paquetes ARP enviados en una LAN, y posteriormente analizar los datos obtenidos. Las mismas fueron escritas en \emph{Python}, utilizando el software para manipulación y análisis de paquetes \emph{Scapy}\footnote{http://www.secdev.org/projects/scapy/}.

Para el desarrollo de este trabajo, realizamos capturas de paquetes ARP sobre cuatro LANs de distinta naturaleza: la red abierta tipo Wi-Fi del \emph{Shopping Alto Palermo}, la red local laboral de la empresa \emph{Honeywell}, la red Wi-Fi \emph{Entrepiso-DC} disponible desde los laboratorios del Departamento de Computación, y la red hogareña de un integrante del grupo.