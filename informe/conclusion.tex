\section{Conclusiones}

La primer conclusión a la que llegamos es que la teoría de la información se condice con la práctica; nos referimos a la relación entre la entropía de una fuente de información y los símbolos distinguidos (en nuestro caso IPs / Tipos de Paquete).\\

Resultó interesante analizar diversas redes, ya que pudimos ver comportamientos y complejidades distintas. Es más fácil analizar lo que pasa en una red hogareña, mientras que la transferencia de información en la red del laboratorio de Computación crece exponencialmente, y es más difícil analizar su estructura sin acceder a otra información.\\

Scapy nos resultó una herramienta fácil de implementar y utilizar, y cumplió su propósito.\\

Llegamos a la conclusión que no siempre los nodos distinguidos se corresponden con los routers de una red, aunque en nuestros casos casi siempre fue así. Sin embargo, habría sido interesante analizar una red ad hoc para evaluar su comportamiento.\\

Finalmente, pudimos observar que en ninguna de las redes analizadas se observa un overhead de paquetes ARP significativo, por ser la proporción de paquetes ARP pequeña respecto a la proporción del resto de los paquetes.

