\section{Desarrollo}

Para analizar el tráfico en las redes, contamos con una fuente de información provista por la cátedra, que identifica los distintos tipos de paquetes capturados.

\begin{itemize}
\item $S=\{s_1 ... s_n\}$ siendo $s_i$ el valor del campo \textit{type} del frame de capa 2.
\end{itemize}

Adicionalmente, se nos pide que propongamos una fuente que nos permita distinguir nodos de una red en base al tráfico ARP. Por tal motivo, los símbolos de dicha fuente tendrían que estar conformados por campos de paquetes ARP.

Creemos que para poder distinguir nodos en una red es necesario obtener el campo $destino$ de un paquete, ya que éste nos permite identificar sobre qué nodos se concentra el tráfico ARP.

Dicho esto, la fuente propuesta es la siguiente:

\begin{itemize}
\item $S=\{s_1 ... s_n\}$ siendo $s_i$ el valor del campo \textit{destination} de los paquetes ARP.
\end{itemize}


Dadas estas fuentes, podemos obtener la \textit{información} del evento ``aparicion del símbolo $s$'' de la siguiente manera:

$I(s) = -log(P(s))$, donde $I(s)$ representa la información obtenida con cada aparición del simbolo $s$, y $P(s)$ la probabilidad de aparición de un simbolo $s$.

Además, podemos calcular la \textit{entropía} de una fuente obteniendo la esperanza de la información de los eventos, como se muestra a continuación:

$H(S) = \sum_{s \in S} P(s) I(s)$

Cabe destacar que la entropía nos da una nocion del grado de incerteza de la fuente, es decir cuanto menor sea el valor de entropía menor será el grado de incerteza. 

Para nosotros, un \textbf{nodo distinguido} va a ser aquel cuya información se encuentre por debajo de la entropía de la fuente. Esto es así pues un nodo con un valor bajo de información, implicará una gran cantidad de apariciones.


