% Configuración del tipo páginas
	\documentclass[11pt, a4paper,english,spanish]{article}
	\parindent = 11pt
	\parskip = 0 pt	
	\usepackage[width=15.5cm, left=3cm, top=2.5cm, height= 24.5cm]{geometry}

% Paquete para reconocer la separación en sílabas en español
	\usepackage[utf8]{inputenc}
	\usepackage[spanish]{babel}

% Paquetes especiales
	\usepackage{amsmath}
	\usepackage{amsthm} %paquete para teoremas y propiedades
	\usepackage{amsfonts}
	\usepackage{amssymb}
	\usepackage[pdftex]{graphicx}
	\usepackage{float}
	\usepackage{graphicx} %paquete para incluir imagenes

	\usepackage{caption}
	\usepackage{subcaption}


\usepackage{algorithm}
\usepackage{algorithmicx}
\usepackage{algpseudocode}
\usepackage{tabularx}
\usepackage{multirow}

% Paquete para incluir hypervinculos
	\usepackage{color}
	\usepackage{url}
	\definecolor{lnk}{rgb}{0,0,0.4}
	%colores para la inclusion del codigo
	\definecolor{gray97}{gray}{.97}
	\definecolor{gray75}{gray}{.75}
	\definecolor{gray45}{gray}{.45}
	\usepackage[colorlinks=true,linkcolor=lnk,citecolor=black,urlcolor=black]{hyperref}

	\newcommand{\todo}{\textbf{\textcolor{red}{TODO: }}}

% Paquete para armar índices
	\usepackage{makeidx}
	\makeindex

\renewcommand{\thefootnote}{\arabic{footnote}} %config de footnote

%Paquete y su config para incluir los codigos de c++
\usepackage{times}
\usepackage{listings}


\lstset{ frame=Ltb,
framerule=0pt,
aboveskip=0.5cm,
framextopmargin=3pt,
framexbottommargin=3pt,
framexleftmargin=0.4cm,
framesep=0pt,
rulesep=.4pt,
backgroundcolor=\color{gray97},
rulesepcolor=\color{black},
tabsize=2,
%
stringstyle=\ttfamily,
showstringspaces = false,
basicstyle=\scriptsize\ttfamily,
commentstyle=\color{gray45},
keywordstyle=\bfseries,
language=C++,
%
numbers=left,
numbersep=15pt,
numberstyle=\tiny,
numberfirstline = false,
breaklines=true,
}

\lstset{emph={%  
    Coord, SolucionEj1, SolucionEj2%
    }, emphstyle={\bfseries}%
}%

% minimizar fragmentado de listados
\lstnewenvironment{listing}[1][]
{\lstset{#1}\pagebreak[0]}{\pagebreak[0]}

\lstdefinestyle{C++}
{language=C++}

\lstdefinestyle{Python}
{language=Python}

	
% Carátula
	\usepackage{caratula}

	%\usepackage[vlined,tworuled,commentsnumbered,linesnumbered]{algorithm2e}

% Más espacio entre líneas
	\parskip=1.5pt

\newcommand{\comment}[1]{\emph{// {#1}\;}} 
\newcommand{\la}{\(\leftarrow\)}
\newcommand{\ra}{\(\rightarrow\)}

\hypersetup{ 
pdfauthor = {},
pdftitle = {}, 
pdfkeywords = {},
pdfstartview={XYZ null null 0.80},
}

\begin{document}
\titulo{TP1}
%\subtitulo{Grupo ???}
%\resumen{
%	\textbf{Resumen: } \\ \\
%	\indent \textbf{Keywords: }
%}
\fecha{Fecha de entrega: 29 de Abril}
\materia{Teoría de las Comunicaciones}
\integrante{Ignacio Gleria}{}{}
\integrante{Patricio Mosse}{515/06}{patricio.mosse@gmail.com}
\integrante{David Temnyk}{779/10}{david.temnyk@hotmail.com}
\integrante{Gustavo Torrecilla}{833/10}{gustavo.d.t\_90@hotmail.com}

\maketitle
% Índice
\newpage \printindex \tableofcontents
\normalsize

\newpage

\section{Introducción}

En este trabajo práctico desarrollamos una herramienta de diagnóstico de red con los objetivos de capturar los paquetes enviados a través de la misma, identificar los distintos protocolos utilizados y, principalmente, analizar el tráfico ARP (\textit{Address Resolution Protocol}) mediante herramientas descriptas en la Teoría de la Información. Este protocolo permite vincular identificadores de la capa de enlace (MAC) con direcciones de la capa de red (IP) de dispositivos conectados a una red local.\\

Para el desarrollo de la herramienta utilizamos el lenguaje de programación Python y el programa de manipulación de paquetes Scapy. La captura de paquetes fue realizada sobre 5 redes: Una red hogareña, una red de un local de Starbucks, una red de un local de McDonald's, una red del laboratorio de computación y una red pública de la línea D de Subte.


\newpage
\section{Desarrollo}

Para analizar el tráfico en las redes, contamos con una fuente de información provista por la cátedra, que identifica los distintos tipos de paquetes capturados.

\begin{itemize}
\item $S=\{s_1 ... s_n\}$ siendo $s_i$ el valor del campo \textit{type} del frame de capa 2.
\end{itemize}

Adicionalmente, se nos pide que propongamos una fuente que nos permita distinguir nodos de una red en base al tráfico ARP. Por tal motivo, los símbolos de dicha fuente tendrían que estar conformados por campos de paquetes ARP.

Creemos que para poder distinguir nodos en una red es necesario obtener el campo $destino$ de un paquete, ya que éste nos permite identificar sobre qué nodos se concentra el tráfico ARP.

Dicho esto, la fuente propuesta es la siguiente:

\begin{itemize}
\item $S=\{s_1 ... s_n\}$ siendo $s_i$ el valor del campo \textit{destination} de los paquetes ARP.
\end{itemize}


Dadas estas fuentes, podemos obtener la \textit{información} del evento ``aparicion del símbolo $s$'' de la siguiente manera:

$I(s) = -log(P(s))$, donde $I(s)$ representa la información obtenida con cada aparición del simbolo $s$, y $P(s)$ la probabilidad de aparición de un simbolo $s$.

Además, podemos calcular la \textit{entropía} de una fuente obteniendo la esperanza de la información de los eventos, como se muestra a continuación:

$H(S) = \sum_{s \in S} P(s) I(s)$

Cabe destacar que la entropía nos da una nocion del grado de incerteza de la fuente, es decir cuanto menor sea el valor de entropía menor será el grado de incerteza. 

Diremos que un \textbf{nodo} o \textbf{protocolo} es \textbf{distinguido} si su información se encuentra por debajo de la entropía de la fuente. Esto es así pues un nodo con un valor bajo de información, implicará una gran cantidad de apariciones.



\newpage
\section{Experimentación}

\subsection{Red Hogareña}

En este experimento, capturamos los paquetes de la LAN de uno de los miembros de nuestro grupo. La medición fue realizada un día sábado desde las 12 hs hasta las 14 hs. La cantidad de paquetes capturados aproximadamente es de 125000. Sin embargo, sólo 153 de estos corresponden al protocolo ARP.

\begin{figure}[H]
       \centering
       \includegraphics[width=0.75\textwidth]{../resultados/Casa/Basename_Source_hist}
       \caption{Protocolos de los paquetes capturados}
       \label{red-hogarena-types}
\end{figure}



\subsection{Red McDonald's}

Para el siguiente experimento, capturamos los paquetes de la LAN Wi-Fi pública del McDonald's ubicado en el shopping Alto Avellaneda. La medición fue realizada un día sábado desde las 18 hs hasta las 20 hs. La cantidad de paquetes capturados es de aproximadamente 65.000. De todos estos, sólo 918 corresponden al protocolo ARP.

\begin{figure}[H]
       \centering
       \includegraphics[width=0.75\textwidth]{../resultados/McDonalds/Basename_Source_hist}
       \caption{Protocolos de los paquetes capturados}
       \label{red-hogarena-types}
\end{figure}



\subsection{Red Starbucks}

\begin{figure}[H]
       \centering
       \includegraphics[width=1\textwidth]{../resultados/Starbucks/histogram_types.png}
       \caption{Protocolos de los paquetes capturados}
       \label{red-Starbucks-types}
\end{figure}

\begin{figure}[H]
       \centering
       \includegraphics[width=1\textwidth]{../resultados/Starbucks/histogram_types_information.png}
       \caption{Información de los protocolos de los paquetes capturados}
       \label{red-Starbucks-types-information}
\end{figure}


\begin{figure}[H]
       \centering
       \includegraphics[width=1\textwidth]{../resultados/Starbucks/histogram_dst.png}
       \caption{IPs destino de los paquetes ARP}
       \label{red-Starbucks-dst}
\end{figure}


\begin{figure}[H]
       \centering
       \includegraphics[width=1\textwidth]{../resultados/Starbucks/histogram_dst_information.png}
       \caption{Información de IPs destino de los paquetes ARP}
       \label{red-Starbucks-dst-information}
\end{figure}


\begin{figure}[H]
       \centering
       \includegraphics[width=1\textwidth]{../resultados/Starbucks/network.png}
       \caption{Tráfico de paquetes ARP}
       \label{red-Starbucks-dst-information}
\end{figure}




\subsection{Red Laboratorios DC}

Para este experimento, capturamos los paquetes de la LAN Wi-Fi Laboratorios-DC del Departamento de Computació de la FCEyN de la UBA. La medición fue realizada un día Lunes desde las 15hs y durante 15 minutos. La cantidad de paquetes capturados es de 4000. De estos, 164 corresponden al protocolo ARP.

\begin{figure}[H]
       \centering
       \includegraphics[width=1\textwidth]{../resultados/labo-corrida3/histogram_types.png}
       \caption{Protocolos de los paquetes capturados}
       \label{red-Starbucks-types}
\end{figure}

\begin{figure}[H]
       \centering
       \includegraphics[width=1\textwidth]{../resultados/labo-corrida3/histogram_types_information.png}
       \caption{Información de los protocolos de los paquetes capturados}
       \label{red-Starbucks-types-information}
\end{figure}

Como podemos observar también en este experimento, de acuerdo a nuestra definición de protocolo distinguido, el protocolo IPv4 sería el único distinguido en esta fuente. Es razonable, ya que la cantidad de paquetes IPv4 es mucho mayor que la cantidad de paquetes IPv6 y ARP. La información de los paquetes IPv4 es \textbf{0.0988917569855}, mientras que la entropía de la fuente es \textbf{0.440025436837}. Se observa como la información es claramente menor a la entropía.

\begin{figure}[H]
       \centering
       \includegraphics[width=1\textwidth]{../resultados/labo-corrida3/histogram_dst.png}
       \caption{IPs destino de los paquetes ARP}
       \label{red-Starbucks-dst}
\end{figure}

\begin{figure}[H]
       \centering
       \includegraphics[width=1\textwidth]{../resultados/labo-corrida3/histogram_dst_information.png}
       \caption{Información de IPs destino de los paquetes ARP}
       \label{red-Starbucks-dst-information}
\end{figure}


\begin{figure}[H]
       \centering
       \includegraphics[width=1\textwidth]{../resultados/labo-corrida3/network.png}
       \caption{Tráfico de paquetes ARP}
       \label{red-Starbucks-dst-information}
\end{figure}

Analizando la información de estos gráficos vemos como la IP \textbf{10.2.203.254} parece ser el router: recibe una mayor cantidad de paquetes que casi todas las demás IPs, es un nodo distinguido por ser su información menor a la entropía, y las conexiones que tiene son a las IPs 10.2.200.X, 10.2.201.X y 10.2.202.X, que deben ser subnets.\\

También vemos por ejemplo que la IP \textbf{10.2.2.230} recibe incluso una mayor cantidad de paquetes y  también es un nodo distinguido, pero esa mayor cantidad de paquetes viene desde pocos nodos. Creemos por esto que puede tratarse de algún servidor (de datos, de imágenes, etcétera).



\newpage
\section{Discusiones}

Algo importante que se pudo observar en casi todas las redes (menos la de los laboratorios del DC), es que existe un nodo (en los grafos) con IP \textit{0.0.0.0} que es fuente de muchos paquetes. En este caso, se trata de paquetes DHCP ARP, los cuales son utilizados para evitar conflictos de IP. Lo que sucede es que cuando un dispositivo quiere ingresar a una red, primero verifica si la direccion IP que se le fue asignada, mediante un protocolo de tipo DHCP, está ocupada por otro dispositivo en la red. Para esto, envía un ARP request desde la dirección IP \textit{0.0.0.0}, a la dirección IP que quiere ocupar, y en caso de no recibir respuesta procede a utilizarla. En caso contrario, se comunica con el servidor DHCP y espera que se le asigne una nueva IP, para luego repetir el procedimiento.

Otra cosa que se pudo observar son nodos que envían paquetes ARP a su misma dirección IP. Estos paquetes son conocidos como GARP (Gratuitous ARP), y se utilizan para actualizar las tablas cache de los demás dispotivos de la red. Por tal motivo este tipo de paquete se envía en broadcast

\newpage
\section{Conclusiones}

La primer conclusión a la que llegamos es que la teoría de la información se condice con la práctica; nos referimos a la relación entre la entropía de una fuente de información y los símbolos distinguidos (en nuestro caso IPs / Tipos de Paquete).\\

Resultó interesante analizar redes disintas ya que pudimos ver comportamiento y complejidades distintas. Es más fácil analizar lo que pasa en una red hogareña, mientras que la transferencia de información en la red del laboratorio de Computación crece exponencialmente muy rápido, y es más difícil analizar su estructura sin acceder a otra información.\\

Scapy nos resultó una herramienta fácil de implementar y utilizar, y cumplió su propósito.\\

Llegamos a la conclusión que no siempre los nodos distinguidos se corresponden con los routers de una red, aunque en nuestros casos casi siempre fue así. Sin embargo, hubiese sido interesante analizar una red ad hoc para evaluar su comportamiento.\\


\end{document}
