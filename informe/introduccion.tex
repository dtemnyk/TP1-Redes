\section{Introducción}

En este trabajo práctico desarrollamos una herramienta de diagnóstico de red con los objetivos de capturar los paquetes enviados a través de la misma, identificar los distintos protocolos utilizados y, principalmente, analizar el tráfico ARP (\textit{Address Resolution Protocol}) mediante herramientas descriptas en la Teoría de la Información. Este protocolo permite vincular identificadores de la capa de enlace (MAC) con direcciones de la capa de red (IP) de dispositivos conectados a una red local.\\

Para el desarrollo de la herramienta utilizamos el lenguaje de programación Python y el programa de manipulación de paquetes Scapy. La captura de paquetes fue realizada sobre 5 redes: Una red hogareña, una red de un local de Starbucks, una red de un local de McDonald's, una red del laboratorio de computación y una red pública de la línea D de Subte.

