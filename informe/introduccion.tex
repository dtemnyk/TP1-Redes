\section{Introducción}

En este trabajo práctico hemos desarrollado una herramienta de diagnóstico de red con el objetivo de capturar los paquetes que se envían a través de la misma, identificar los distintos protocolos utilizados y principalmente, analizar el tráfico ARP (\textit{Address Resolution Protocol}) mediante herramientas provistas por la teoría de la información. Este protocolo permite vincular identificadores de la capa de enlace (MAC) con direcciones de la capa de red (IP) de dispositivos conectados a una red local.


Hemos utilizado el lenguaje de programación Python para el desarrollo de la herramienta, valiéndonos principalmente del software de manipulación de paquetes Scapy. La captura de paquetes fue realizada sobre 4 redes que... \textbf{TODO:} Definir cuáles van a ser...

