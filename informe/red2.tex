\subsection{Red McDonald's}

Para el siguiente experimento, capturamos los paquetes de la LAN Wi-Fi pública del McDonald's ubicado en el shopping Alto Avellaneda. La medición fue realizada un día sábado desde las 18 hs hasta las 20 hs. La cantidad de paquetes capturados es de aproximadamente 65.000. De todos estos, sólo 918 corresponden al protocolo ARP.

\begin{figure}[H]
       \centering
       \includegraphics[width=1\textwidth]{../resultados/McDonalds/histogram_types.png}
       \caption{Protocolos de los paquetes capturados}
       \label{red-hogarena-types}
\end{figure}

En este caso el overhead impuesto por los paquetes ARP es de 1.411656\%

\begin{figure}[H]
       \centering
       \includegraphics[width=1\textwidth]{../resultados/McDonalds/histogram_types_information.png}
       \caption{Información de los protocolos de los paquetes capturados}
       \label{red-hogarena-types}
\end{figure}

Como podemos observar, de acuerdo a la definición de protocolo distinguido que dimos anteriormente, el protocolo IPv4 sería el único distinguido en esta fuente. Esto resulta razonable, ya que la cantidad de paquetes IPv4 es mucho mayor que la cantidad de paquetes IPv6, LLC y ARP. La información de los paquetes IPv4 es \textbf{0.118172764791}, mientras que la entropía de la fuente es \textbf{0.51288679091}. Se observa claramente como la información es menor a la entropía.

\begin{figure}[H]
       \centering
       \includegraphics[width=1\textwidth]{../resultados/McDonalds/histogram_dst.png}
       \caption{IPs destino de los paquetes ARP}
       \label{red-hogarena-arp-destination}
\end{figure}

\begin{figure}[H]
       \centering
       \includegraphics[width=1\textwidth]{../resultados/McDonalds/histogram_dst_information.png}
       \caption{Información de IPs destino de los paquetes ARP}
       \label{red-hogarena-arp-destination-info}
\end{figure}

En este caso el valor de la entropía es de \textbf{5.22269613542}, y se pueden observar ocho IPs que se encuentran debajo de este valor, por lo que consideramos que son nodos distinguidos en la red. Dichas IPs son:
\begin{itemize}
\item IP: \textit{192.168.2.1} con valor de información de \textbf{4.16992500144}
\item IP: \textit{172.17.8.1} con valor de información de \textbf{2.57556380272}
\item IP: \textit{192.168.0.1} con valor de información de \textbf{4.45003292064}
\item IP: \textit{169.254.70.115} con valor de información de \textbf{4.88815403303}
\item IP: \textit{169.254.123.32} con valor de información de \textbf{3.41608558871}
\item IP: \textit{17.173.254.222} con valor de información de \textbf{4.79795622406}
\item IP: \textit{17.173.254.223} con valor de información de \textbf{4.63289697778}
\item IP: \textit{169.254.137.106} con valor de información de \textbf{3.41608558871}
\end{itemize}

Es importante destacar que el nodos con IP:\textit{172.17.8.1} tiene una información mucho menor al resto, por lo que probablemente se trate de un router.

A continuación se muestra el tráfico de la red para poder reconocer con qué dispositivos se condicen los \textbf{nodos distinguidos}.

\begin{figure}[H]
       \centering
       \includegraphics[width=1\textwidth]{../resultados/McDonalds/network.png}
       \caption{Tráfico de paquetes ARP}
       \label{red-hogarena-arp-traffic}
\end{figure}

En este caso se observa que el nodo con IP: \textit{172.17.98.1}, es destino de varios nodos, por lo que confirma aun más nuestra hipotesis de que se trate de un router.

Por otro lado, es inevitable observar que el nodo con IP: \textit{0.0.0.0} es fuente de una gran cantidad de paquetes. Si bien hablaremos más de esto en la sección \textit{Discusión}, por el momento vamos a adelantar que esta dirección es utilizada inicialmente por dispositivos que se conectan a la red y buscan saber si una IP determinada se encuentra o no disponible para utilizarla. 

En cuanto al resto de los nodos distinguidos, muchos de los paquetes son recibidos desde la dirección IP: \textit{0.0.0.0}. Creemos que lo que puede estar sucediendo en este caso es que distintos dispositivos se conectaban a la red buscando utilizar estas IPs, y una vez que se desconectaban de la red, aparecían nuevos dispositivos con la intención de utilizar estas mismas direcciones IP. Por tal motivo, la cantidad de paquetes ARP enviados a las direcciones IP correspondientes a los nodos distinguidos resulta ser elevada.
