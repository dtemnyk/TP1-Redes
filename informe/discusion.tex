\section{Discusiones}

Algo importante que se pudo observar en casi todas las redes (menos la de los laboratorios del DC), es que existe un nodo (en los grafos) con IP \textit{0.0.0.0} que es fuente de muchos paquetes. En este caso, se trata de paquetes DHCP ARP, los cuales son utilizados para evitar conflictos de IP. Lo que sucede es que cuando un dispositivo quiere ingresar a una red, primero verifica si la direccion IP que se le fue asignada mediante un protocolo de tipo DHCP, está ocupada por otro dispositivo en la red. Para esto, envía un ARP request desde la dirección IP \textit{0.0.0.0}, a la dirección IP que quiere ocupar, y en caso de no recibir respuesta procede a utilizarla. En caso contrario, se comunica con el servidor DHCP y espera que se le asigne una nueva IP, para luego repetir el procedimiento.\\

Otra cosa que se pudo observar son nodos que envían paquetes ARP a su misma dirección IP. Estos paquetes son conocidos como GARP (Gratuitous ARP), y se utilizan para actualizar las tablas caché de los demás dispotivos de la red. Por tal motivo este tipo de paquete se envía en broadcast.
